\documentclass{article}
\usepackage{tikz-cd}
\usepackage{lindrew}
\title{Lecture 5}
\date{January 28, 2025}
\author{Muna Naik}

\newcommand*{\slrg}[1]{SL_{#1 \times #1}(\mathbb{R})}
\newcommand*{\slcg}[1]{SL_{#1 \times #1}(\mathbb{C})}
\newcommand*{\glrg}[1]{GL_{#1 \times #1}(\mathbb{R})}
\newcommand*{\glcg}[1]{GL_{#1 \times #1}(\mathbb{C})}
\newcommand*{\mr}[1]{M_{#1 \times #1}(\mathbb{R})}
\newcommand*{\org}[1]{O_{#1 \times #1}(\mathbb{R})}

\begin{document}

\maketitle

\section{The exponential map}

\begin{definition}
    Let $G$ be a lie group and let $\phi:\RR\to G$ be a smooth group homomorphism. Then $\phi$ is called a \vocab{one-parameter subgroup} of $G$.
\end{definition}

Given a vector in $T_e(G)$ let $\tilde{X}\in \mathcal{D}_L^1(G)$ denote the left invariant vector field corresponding to it and $\gamma_{\tilde{X}}$ the integral curve of it such that $\gamma_{\tilde{X}}(0)=e$. Then proposition 5 of the last lecture tells us that
$$\gamma_{\tilde{X}}(s+t) = \gamma_{\tilde{X}}(s)\cdot \gamma_{\tilde{X}}(t)$$
Also note that if $\eta:\RR\to G$ is given by $\eta(t) = \gamma_{\tilde{X}}(st)$ then $\eta(0)=e$ and it is an integral curve of $s\tilde{X}$. By uniqueness of integral curves we have,
$$\gamma_{\tilde{X}}(st) = \gamma_{s\tilde{X}}(t)$$

\begin{definition}
    Let $X\in \mathfrak{g}$. Then we define the \vocab{exponential} of $X$ as, $$\exp(X) = \gamma_{\tilde{X}}(1)\in G$$
\end{definition}

\begin{proposition}
    The following properties hold for the exponential map:
    \begin{enumerate}
        \item $\exp(0) = e$ 
        \item $\gamma_{\tilde{X}}(s) = \exp(sX)$
        \item $\exp((s+t)X) = \exp(sX)\exp(tX)$
        \item $\eta(t) = \exp(tX)$ is the integral curve of $\tilde{X}$ with $\eta(0)=e$. 
        \item Fix $g\in G$. $\alpha(t) = g\cdot \exp(tX)$ is the integral curve of $\tilde{X}$ with $\alpha(0)=g$.
    \end{enumerate}
\end{proposition}

\begin{proof}
    They straightforward from the observations above.
    \begin{enumerate}
        \item $exp(0) = \gamma_0(1) = e$ since zero vector field has constant functions as integral curves.
        \item Note that $s\tilde{X} = \widetilde{sX}$. Thus $$\gamma_{\tilde{X}}(s) = \gamma_{s\tilde{X}}(1) = \gamma_{\widetilde{sX}}(1) = exp(sX)$$
        \item Using the property of integral curves above, we have $$\gamma_{\tilde{X}}(st) = \gamma_{s\tilde{X}}(t)$$
        $$\implies exp((s+t)X) = exp(sX)exp(tX)$$
        \item It follows from property 2.
        \item This follows from the first observation about left invariant vector fields from section 3 of lecture 4.
    \end{enumerate}
\end{proof}
 
\begin{example*}
    \begin{enumerate}
        \item $(\RR^n,+)$ has trivial lie algebra. The integral curves are easily seen to be $\gamma(t) = tv$. Thus the exponential map is the identity.
        \item $(\RR^*,\times)$ has lie algebra $\RR$. The left invariant vector field is $X^a_p = dl_p(a) = pa$. The integral curves is easily seen to be $\gamma(t) = e^{at}$. Thus $\exp(a) = e^a$.
        \item Let $G=S^1$. clearly $\mathfrak{g} = \RR$. Show that $exp(t) = e^{it}$(Exercise).
        \begin{proof}
            Consider $S^1$ as being embedded in $\CC$ as the unit circle. Then the left translation of the group corresponds to rotation ofthe complex plane which is a linear map and hence its own derivative. So, say $X$ is a tangent vector to $S^1$ at $1$. Then $\tilde{X}_{e^{i\theta}} = dl_{e^{i\theta}}(X)$ which is essentially $X$ rotated by $\theta$ counterclockwise. Identify the tangent vectors with complex numbers via translation to origin. Then $\tilde{X}_{e^{i\theta}} = ie^{i\theta} = (e^{it})'(\theta)$
            Thus $\tilde{X}_{\gamma(\theta)} = \gamma'(\theta)$ if $\gamma(t) = e^{it}$. Hence it is the integral curve by uniqueness. 
        \end{proof}
        \item Let $G=GL_n(\RR)$. Then $\mathfrak{g} = M_n(\RR)$. Let $X\in \mathfrak{g}$. Then $exp(X) = e^X$ where $e^X$ is the matrix exponential.(Prove this!)
        \begin{proof}
            Let $X\in M_n(\RR)$. Then let $\gamma(t): e^(tX) = \sum_{n=0}^{\infty} \frac{t^nX^n}{n!}$. Then we have $$\gamma'(t) = \lim_{h\to 0}\frac{e^{(t+h)X}-e^tX}{h} = \lim_{h\to 0}e^{tX}\frac{e^{hX}-1}{h} = e^{tX}X = d(L_{e^{tX}})X = \tilde{X}_{\gamma(t)}$$
            Thus $\gamma$ is an integral curve of $\tilde{X}$ with $\gamma'(0)=X$. By uniqueness of integral curves we have $$e^X = \gamma(1) = \gamma_{\tilde{X}}(1) = exp(X)$$
            As was desired.
        \end{proof}
    \end{enumerate}
    
\end{example*}

\section{Properties of the exponential map}

\begin{proposition}
    The $exp$ map is smooth from $\mathfrak{g}$ to $G$. Further, its derivative at zero is the identity map.
    Thus by the inverse function theorem, $exp$ is a local diffeomorphism around zero.
\end{proposition}

\begin{proof}
    Consider $H = \mathfrak{g}\times G$. Define $V\in \mc{D}^1(H)$ as, $V_{(X,g)} = (0,\tilde{X}_g)$. Let $\eta_{(X,g)}:\RR\to H$ be given by 
    $$\eta_{(X,g)}(t) = (X,gexp(tX))$$
    Then $\eta_{(X,g)}$ is an integral curve of $V$ with $\eta_{(X,g)}(0) = (X,g)$. Let $F$ be the flow of this vector field as in theorem 2 of lecture 4. Consider the maps
    $$\mathfrak{g}\to \RR\times \mathfrak{g}\times G\to \mathfrak{g}\times G\to G$$
    $$X\mapsto (1,X,e)\mapsto F(1,X,e)\mapsto \pi_2(F(1,X,e))$$
    Since the flow is smooth, the composition is smooth. Note that $\pi_2(F(1,X,e)) = \pi_2(\eta_{(X,e)}(1)) = \pi_2((X,exp(X))) = exp(X) $. So $exp(X)$ is smooth.
    Now observe that $$[exp'(0)](X) = (exp(tX))'(0) = \gamma_{\tilde{X}}'(0) = X$$
    Since $X$ was arbitrary the derivative at zero is the identity map.
\end{proof}

\begin{lemma}
    Let $\alpha:\RR\to G$ be a one parameter subgroup. Suppose $\alpha'(0) = X$. Then $$\alpha(t) = exp(tX)$$
\end{lemma}
\begin{proof}
    We have $\alpha'(t) = \frac{d}{ds}\alpha(t+s)|_{s=0} = \frac{d}{ds}\alpha(t)\alpha(s)|_{s=0} = (dl_{\alpha(t)})_e(X) = \tilde{X}_{\alpha(t)}$. This shows that $\alpha$ is an integral curve. By uniqueness of integral curves we have $\alpha(t) = \gamma_{\tilde{X}}(t) = exp(tX)$.
\end{proof}

Thus one parameter subgroups of $G$ are precisely the integral curves of left invariant vector fields. Morover they are uniquely determined by their derivative at zero.
We will use this to show the following theorem which is very handy in manipulating the exponential map.

\begin{theorem}
    Let \mm{\phi:G\to H} be a lie group homomorphism. Then \mm{d\phi_e:\mathfrak{g}\to \mathfrak{h}} is a lie algebra homomorphism(Lecture 3 proposition 4).
    And we have the following commutative diagram.
    \begin{center}
        \begin{tikzcd}
            \mathfrak{g} \arrow[r,"exp"] \arrow[d,"d\phi_e"] & G \arrow[d,"\phi"] \\
            \mathfrak{h} \arrow[r,"exp"] & H
        \end{tikzcd}
    \end{center}
    In other words, \mm{\phi\circ exp = exp\circ d\phi_e}.
\end{theorem}

\begin{proof}
    Let $X\in \mathfrak{g}$. Then $\gamma(t):=\phi(exp(tX))$ is a smooth map. And being a composition of group homomorphisms is a group homomorphism itself from $\RR \to H$. Thus it is a one parameter subgroup of $H$
    with $\gamma'(0) = d\phi_e(X)$(By chain rule). By the lemma above we immediately get $$\phi(exp(tX)) = \gamma(t) = exp(t\gamma'(0)) = exp(td\phi(X))$$
    Put $t=1$ to get the required.
\end{proof}

\section*{Topological groups}

We will digressa bit to discuss certain properties of topological groups that will be usefull later.
Suppose $G$ is a topological group. Which is a group equipped with a topology that is compatible with its group operations. 
We will denote by $G_0$ the connected component of the identity. Observe that $G_0$ is always a closed.

\begin{exercise*}
    Show that $G_0$ is a normal subgroup of $G$.
\end{exercise*}

\begin{proof}
    Let $g\in G_0$. Then we know that $l_g:G\to G$ defined by left multiplication is a continuous map. Thus $l_g(G_0)$ is connected. And since $e\in G_0,l_g(e)=g$ we have $G_0\ni g\in l_g(G_0)\implies l_g(G_0)\cap G_0\neq \phi$.
    Since $G_0$ is a connected component by maximality we have $l_g(G_0)\subseteq G_0$. This shows that $G_0$ is closed under multiplication. Note that inverse is also a continuous map and $e^{-1}=e\implies G_0^{-1}\cap G_0\neq \phi$.
    so again by maximality of connected components we get $G_0^{-1}\subseteq G_0$. Thus we have that $G_0$ is a subgroup of $G$.\\
    Let $h\in G$. Then the map $c_h:g\mapsto hgh^{-1}$ is a continuous map as it is the composition of the continuous maps $l_h,r_{h^{-1}}$. And $c_h(e)=e\implies G_0\ni e\in c_h(G_0)$
    Thus $c_h(G_0)\cap G_0\neq \phi$. By maximality of connected components we have $c_h(G_0)\subseteq G_0$. Thus $G_0$ is normal.
\end{proof}

\begin{exercise*}
    Say $H$ is an open subgroup of $G$. Show that $H$ is closed.
\end{exercise*}

\begin{proof}
    Say suppose $x\in G-H$. Then if $y\in l_x(H)\cap H$, then $y=xh$ for some $h\in H$. But then $x=yh^{-1}\in H$ which is a contrdiction. Thus $l_x(H)\cap H=\phi$(It is afterall the left coset containing $x$). 
    But we observe that $l_x$ is a homeomorphism as it is continuous and has $l_{x^{-1}}$ as its continuous inverse. Thus $l_x(H)$ is open. So we have funished an open neighbourhood of $x\in G-H$ in $G-H\implies G-H$ is open
    $\implies H$ is closed. 
\end{proof}

\begin{definition}
    Let $U$ be an open neighbourhood of $e$. Then $U$ is said to be a \vocab{symmetric neighbourhood} of $e$ if $U=U^{-1}$.
\end{definition}

\begin{exercise*}
    Show that any given neighbourhood $U$ of $e$ contains a symmetric neighbourhood $V$ of $e$ such that $V\subseteq U,V^2\subseteq U$.
\end{exercise*}
\begin{proof}
    We know that $m:G\times G\to G$ given by $(x,y)\mapsto x\cdot y$ is continuous. Thus the set $m^{-1}(U)$ is open in $G\times G$. $e\in U\implies (e,e)\in m^{-1}(U)$. By definition of product topology we have that there are $V_1,V_2$ open in $G$ such that 
    $V_1\times V_2\subseteq m^{-1}(U)$ and $(e,e)\in V_1\times V_2$. Now set $\tilde{V} = V_1\cap V_2$. Then $\tilde{V}^2 \subseteq V_1\times V_2\subseteq U$. To get a symmetric neighbourhood we set $V = \tilde{V}\cap \tilde{V}^{-1}$. As inverse operation is a 
    homeomorphism it is clear that $V$ is open.
\end{proof}

For any given $S\subseteq G$ let $\langle S\rangle$ denote that subgroup generated by $S$. 

\begin{exercise*}
    If $U$ is open then $\langle U\rangle$ is an open subgroup of $G$. If $U$ is symmetric then $\langle U\rangle = \bigcup_{n\in \NN}U^n$.
\end{exercise*}
\begin{proof}
    Suppose $x\in \langle U\rangle$. Fix $y\in U$. Then $xy^{-1}U$ is an open set contained in $\langle U\rangle$ and it is open since $g\mapsto xy^{-1}g$ is a homeomorphism of $G$.
    Thus $\langle U\rangle$ is open. It is clear that $\bigcup_{n\in \NN}U^n\subseteq \langle U\rangle$. Note that $a\in U^m,b\in U^n\implies ab^{-1}\in U^{m+n}$ since $U$ is symmetric. 
    Thus $\bigcup_{n\in \NN}U^n$ is a subgroup of $G$ containing $U$. Thus $\langle U\rangle \subseteq \bigcup_{n\in \NN}U^n$ as was required.
\end{proof}


\end{document}