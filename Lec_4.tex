\documentclass{article}
\usepackage{tikz-cd}
\usepackage{lindrew}
\title{Lecture 4}
\date{January 21, 2025}
\author{Muna Naik}

\newcommand*{\slrg}[1]{SL_{#1 \times #1}(\mathbb{R})}
\newcommand*{\slcg}[1]{SL_{#1 \times #1}(\mathbb{C})}
\newcommand*{\glrg}[1]{GL_{#1 \times #1}(\mathbb{R})}
\newcommand*{\glcg}[1]{GL_{#1 \times #1}(\mathbb{C})}
\newcommand*{\mr}[1]{M_{#1 \times #1}(\mathbb{R})}
\newcommand*{\org}[1]{O_{#1 \times #1}(\mathbb{R})}

\begin{document}

\maketitle

\begin{question*}
    Do all topological groups admit a lie group structure?
\end{question*}

This is a very deep question. It was one of Hilbert's problems for the turn of the century. 
There is a book by \textit{Montgomery and Zippin} which deals with this question.

\section{Recap of ODEs}

Consider the IVP 
$$y'(t)=g(t,y(t))\text{ and }y(t_0)=p$$
Where \mm{g:\Omega\to \mathbb{R}^n} where \mm{\Omega} is an open subset of \mm{\mathbb{R}\times \mathbb{R}^n} and \mm{(t_0,p)\in \Omega}. 
Then we have the following fundamental result.

\begin{theorem}
    If \mm{g} is continuous and lipschitz in the second variable, then the IVP has a unique local solution.
\end{theorem}

One also has examples like $g(t,y) = 3y^{2/3}$ which have infinitely many solutions. Another natural question of importance is that of the dependence 
of the value of the solution on slight perturbations of the initial value. We gather all the necessary results into one fundamental theorem.

\begin{theorem}
    Let \mm{I\subseteq \mathbb{R}} be open and \mm{V\subseteq\mathbb{R}^n} be open and let \mm{t_0\in I}
    and \mm{p\in U}. Let \mm{g: I\times V\to \mathbb{R}^n} be a \mm{C^k} map. Then there is some \mm{\epsilon>0} such that \mm{(t_0-\epsilon,t_0+\epsilon)\subseteq I} and a neighbourhood \mm{V} of \mm{p} in \mm{U} and 
    a map \mm{F: (t_0-\epsilon,t_0+\epsilon)\times V\to U} such that,
    \begin{enumerate}
        \item \mm{\frac{d}{dt}F(t,x) = g(t,F(t,x))}
        \item \mm{F(t_0,x) = x}
    \end{enumerate}
\end{theorem}

\newpage

\section{Integral curves}

\begin{definition}
    Let \mm{M} be a smooth manifold and \mm{X\in \mc{D}^{1}(M)}. A smooth curve \mm{\gamma:I\to M} is called an integral curve of \mm{X} if,
    $$\dot{\gamma}(t) = X_{\gamma(t)}$$
    If \mm{q\in \gamma(I)} then we say \mm{\gamma} is an integral curve through \mm{q}.
\end{definition}

\begin{exercise*}
    Let \mm{\gamma:I\to M} be an integral curve of \mm{X}. Let \mm{t_0\in I}. Then show that,
    \begin{itemize}
        \item[(A)] Let \mm{J = I - t_0} and let \mm{\delta:J\to M} be defined by \mm{\eta(t) = \gamma(t+t_0)}. Then \mm{\eta} is an integral curve of \mm{X}.
        \item[(B)] Let \mm{K = \frac{1}{s}I} and let \mm{\zeta:K\to M} be defined by \mm{\zeta(t) = \gamma(st)}. Then \mm{\zeta} is an integral curve of \mm{sX}. 
    \end{itemize}
\end{exercise*}

Let \mm{X\in \mc{D}^{1}(M)} and let \mm{p\in M}. Let \mm{\phi = (\phi_1,\phi_2,..,\phi_n)} be a centered co-ordinate chart around \mm{p} valid in a neighbourhood \mm{U}.
let \mm{\gamma:I\to M} be a smooth curve where \mm{I} is a neighbourhood containing \mm{t_0} such that \mm{\gamma(t_0)=p}. Since \mm{\gamma} is continuous, we can select a neighbourhood \mm{J} of \mm{t_0} such that \mm{\gamma(J)\subseteq U}. 
On \mm{J}, we have the following.
\begin{eqnarray*}
    \gamma'(t) = X_{\gamma(t)} &\iff& d\phi_{\gamma(t)} \gamma'(t) = d\phi_{\gamma(t)}X_{\gamma(t)}\\
    &\iff& (\phi\circ \gamma)'(t) = d\phi_{\gamma(t)}X_{\gamma(t)}\\
    &\iff& (\phi\circ \gamma)'(t) = d\phi_{\phi^{-1}\circ(\phi\circ\gamma(t))}X_{\phi^{-1}\circ\phi\gamma(t)}\\
    &\iff& \phi\circ \gamma\text{ is an integral curve of }\tilde{X}_q=d\phi_{\phi^{-1}(q)}X_{\phi^{-1}(q)}\text{ for }q\in \phi(U)
\end{eqnarray*}

So now the fundamental theorem applies and we get a smooth function $F:(-\epsilon,\epsilon)\times V\to M $ where \mm{V} is some neighbourhood of \mm{p}, satisfying the following
$$\frac{d}{dt}F(t,q) = X_{F(t,q)}\,\forall q\in V$$
This function is called the \textbf{local flow} of \mm{X}.

\begin{proposition}
    Let \mm{\gamma,\eta:I\to M} be integral curves of \mm{X} such that \mm{\gamma(t_0) = \eta(t_0) = p}. Then \mm{\gamma = \eta}
\end{proposition}
\begin{proof}
    Let \mm{J = \{t\in I: \gamma(t) = \eta(t)\}}. Then \mm{J} is non-empty and open(by the fundamental theorem and unique local solutions) and closed by continuity.
    But since intervals are connected, we have that \mm{J = I}. As was desired.
\end{proof}

The proposition gives us the fact that there is a maximal interval(\mm{I_p}) given by all intervals on which integral curves through \mm{p} are defined, on which the integral curves exists.
If \mm{I_p = \mathbb{R}\forall p\in X} then we say that the vector field \mm{X} is \vocab{complete}.

\begin{exercise*}
    Show that on a compact manifold, every vector field is complete.
\end{exercise*}
\begin{proof}
    Suppose we have a solution \mm{\gamma:I\to M} where $I=(a,b)$. This always exists by the discussion above. We will first show that the solution extends to \mm{b} continuously.
    Pick a sequence of local parametrizarions \mm{\{\phi_i\}_{1\leq i\leq N}} with the following properties.
    \begin{enumerate}
        \item $\phi_i:B(0,1)\to M$ and they all extend smoothly to the boundary.
        \item $\{\phi_i(B(0,1/2))\}_{1\leq i\leq N}$ is a cover of \mm{M} 
    \end{enumerate}
    This is possible only because of compactness. Now, pull back the vector field on the manifold on each of the co-ordinate balls and consider their sup norm.
    This does exist finitely since the field extends to the closure of the balls. Suppose the maximum of these norms is \mm{M_0}. Now consider \mm{x\in (a,b)} such that 
    \mm{(|b-x|<\frac{1}{4M_0})}. Suppose \mm{\gamma(x)\in \phi_k(B(0,1/2))}. Then we would have that for all $x\leq y<b$, $$\|\phi^{-1}(\gamma(x))-\phi^{-1}(\gamma(y))\|\leq \frac{1}{4M_0}M_0 \leq \frac{1}{4}$$
    Hence $\gamma([x,b))$ is entirely contained in \mm{\phi_k(B(0,1))}. Also since we have the inequality, 
    $$\|\phi^{-1}(\gamma(a))-\phi^{-1}(\gamma(b))\|\leq M_0|a-b|$$ 
    we have that all sequences going to \mm{b} are cauchy, and thus by completeness, must converge to some point in \mm{M}. By continuity of the vector field, it is clear that 
    one side derivative of this extension is the vector field itself. Thus by applying local existence theorem at \mm{\gamma(b)} we can extend the solution beyond \mm{b}.
    Since this can be done for any \mm{b} we have that the interval of existence must be infinite to the left. The same argument can be applied at the other end.
\end{proof}

\begin{example*}
    Look for more examples in Kumaresan's book.
    \begin{enumerate}
        \item On \mm{U\subseteq \mathbb{R}^n} define \mm{X = \frac{\partial}{\partial u_k}}. then the integral curves are $$\gamma_p(t) = \vec{p}+t\vec{e}_k$$
        \item Let \mm{M=\mathbb{R^2}},\mm{Z = x\frac{\partial}{\partial x}+y\frac{\partial}{\partial y}}. Then via \mm{p=(a,b)} the curve 
        $$\gamma_p(t) = (e^ta,e^tb)$$
        is an integral curve of \mm{Z}. This also shows that the vector field is complete.
        \item Let \mm{M = \mathbb{R}^2} and \mm{Z = e^{-x}\frac{\partial}{\partial x}+ y\frac{\partial}{\partial y}}. Through \mm{p=(a,b)} the integral curves turn out to be,
        $$\gamma_t(p) = (log(t+e^a),be^t)$$
        This shows that the vector field is not complete.
    \end{enumerate}
\end{example*}

\section{Back to Lie groups}

Let \mm{G} be a lie group and \mm{X\in \mc{D}_L^1(G)}. Let \mm{\gamma:I\to G} be an integral curve of \mm{X}. Let \mm{a\in G}. Define, \mm{\eta:I\to G} by \mm{\eta(t) = a\gamma(t)}. Then we have,
\begin{eqnarray*}
    \eta'(t) &=& (d\eta)_t\left( \frac{d}{dt} \right)\\
    &=& (dl_a)_{\gamma(t)}(d\gamma)_t\left( \frac{d}{dt} \right)\\
    &=& (dl_a)_{\gamma(t)}X_{\gamma(t)}\\
    &=& X_{a\gamma(t)}\\
\end{eqnarray*}

Hence \mm{\eta} is an integral curve of \mm{X}. This shows that the left translation of an integral curve is also an integral curve.

\begin{claim*}
    Left invariant vector fields are complete.
\end{claim*}
\begin{proof}
    In view of the discussion above, it suffices to show that integral curves through the identity are defined on all of \mm{\mathbb{R}}.
    From the local existence theorem, assume that we have an integral curve \mm{\gamma:(-\epsilon,\epsilon)\to G} such that $\gamma(0)=e$. Pick $0<\tau<\epsilon$.
    Then consider $\tilde{\gamma}:(\tau-\epsilon,\tau+\epsilon)\to G$ given by $\tilde{\gamma}(t) = \gamma(\tau)\gamma(t-\tau)$.  Then note that, 
    $$\tilde{\gamma}'(\tau) = dl_{\gamma(\tau)}\gamma'(0) = \gamma'(\tau)$$
    $\tilde{\gamma}$ will be an integral curve by the discussion above, and the argument above shows that it agrees with \mm{\gamma} at \mm{\tau}. Thus, it must extend the solution to $(-\epsilon,\tau+\epsilon)$
    by uniqueness of integral curves. Repeating the procedure indefenitely, we can extend the solution to all of \mm{\mathbb{R}}.
\end{proof}

Also observe that this means $\gamma(t+s) = \gamma(t)\gamma(s)$. Hence \mm{\gamma} is a group homomorphism from \mm{\mathbb{R}} to \mm{G}.

\end{document}