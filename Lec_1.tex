\documentclass{article}
\usepackage{tikz-cd}
\usepackage{lindrew}
\title{Lecture 1}
\date{January 7, 2025}
\author{Muna Naik}

\newcommand*{\slrg}[1]{SL_{#1 \times #1}(\mathbb{R})}
\newcommand*{\slcg}[1]{SL_{#1 \times #1}(\mathbb{C})}
\newcommand*{\glrg}[1]{GL_{#1 \times #1}(\mathbb{R})}
\newcommand*{\glcg}[1]{GL_{#1 \times #1}(\mathbb{C})}
\newcommand*{\mr}[1]{M_{#1 \times #1}(\mathbb{R})}

\begin{document}

\maketitle

\section*{Logistics}

\begin{itemize}
    \item Grading policy will be as follows. $40\%$-Final exam, $35\%$ - Presentations and quizzes, $25\%$ - Scribing. No midterms!
    \item We will follow several books as listed below,
    \begin{enumerate}
        \item \textit{Kumaresan, A Course in differential geometry and lie groups.}(Ideal for beginners)
        \item \textit{Sigurõur Helgarson, Differential geometry, Lie groups and Symmetric spaces.}(A bit advanced)
        \item \textit{Claude Chevalley, Theory of Lie groups.}(A classic)
    \end{enumerate} 
\end{itemize}

\section{Preliminaries}

A course on differential geometry is highly recommended before this course. However, we will recall certain definitions and results from differential geometry 
in this section. 

\begin{definition}
    Suppose 
    \begin{enumerate}
        \item A \vocab{tangent vector} is a linear map \mm{v:C^\infty (M)\to \RR} satisfying the leibnitz rule.
        $$v(fg) = f(p)v(g)+g(p)v(f)$$
        The collection of these is called as the tangent space at \mm{p} denoted as $T_p(M)$
        \item Say \mm{\phi:M\to N} is a smooth map. Then we can define for each \mm{p\in M} a map $d\phi:T_p(M)\to T_{\phi(p)}(N)$
        called its \vocab{derivative} as,
        $$[d\phi_p(v)](f) = v(f\circ \phi)$$
    \end{enumerate}
\end{definition}

One can check that tangent spaces are finite dimensional, and that this derivative map is a linear map and so on and so forth.
Now we shall define certain special subsets of a manifold.

\begin{definition}
    Let \mm{M,N} be smooth manifolds and let \mm{\phi:M\to N} be a smooth map.
    Then, \mm{\phi} is said to be an \vocab{immersion} or a \vocab{submersion} at \mm{p} when \mm{d\phi_p}
    is injective or surjective respectively. If a map is an immersion or a submersion at all points the map is said to be 
    an immersion or submersion respectively.
\end{definition}

\begin{definition}
    Let \mm{N} be a manifold. A pair \mm{(i,M)} is called a \vocab{submanifold} of \mm{N} if,
    \begin{enumerate}
        \item \mm{M} is a (smooth)manifold.
        \item \mm{i:M\to N} is a one-one smooth immersion.
    \end{enumerate}
\end{definition}

Note that in the above setting, \mm{i} defines a topology on \mm{i(M)}. This we shall call as the \vocab{map topology}.
As \mm{i} is continuous, the subspace topology is coarser than the map topoogy.

\begin{definition}
    When subspace topology is the map topology, the sub manifold is said to be embedded.
\end{definition}

\begin{exercise*}
    If \mm{M} is compact, then any 1-1 immersion of it will give an embedded manifold.
\end{exercise*}
\begin{proof}
    Is is elementary topology that a continuous injection from a compact space to a housdorff space is a homeomorphism onto the image.
    Hence, the subspace topology is the map topology.
\end{proof}

We also have an alternate characterization of submanifolds as given by the following lemma.

\begin{lemma}
    Let \mm{(\phi,N)} be a submanifold of \mm{M}. Then this gives a unique manifold structure on \mm{\phi(N)} that makes 
    \mm{\phi} a diffeomorphism onto its image.
\end{lemma}

Using this lemma we can give the following definition.

\begin{definition}
    Let \mm{N} be a manifold. \mm{M} is called a submanifold of \mm{N} is 
    \begin{enumerate}
        \item \mm{M} is a manifold and \mm{M\subseteq N}
        \item The inclusion map is a smooth immersion.
    \end{enumerate}
\end{definition}

\begin{example}
    \mm{\RR} is an embedded manifold of \mm{\RR^2}.
\end{example}
\begin{example}
    Consider \mm{\phi:\RR\to \RR^2} given by \mm{\phi(t) = \left(\frac{2t}{1+t^2},\frac{t}{\sqrt{1+t^2}}\right)}. This can be verified to be an immersion as it is injective and the derivative doesn't vanish.
    But it is not an embedded submanifold for the follwing reason. The image of say \mm{(-1,1)} is open in the map topology by its very definition.
    But in the subspace topology, this cannot be open, as all neighbour hoods of origin will have preimages of arbitrarily large maginitude in \mm{\RR} under \mm{\phi}.
\end{example}

%% plot the curve of \phi map defined above. using the appropriate packages

We recall two theorems about submanifolds that we will be using later.

\begin{theorem}
    Say \mm{M} and \mm{N} are two manifolds, and let \mm{S\subset N} be a submanifold. Let 
    \mm{\phi:M\to N} be a smooth map and \mm{\phi(M)\subseteq S}. Then, \mm{\tilde{\phi}:M\to S} is smooth iff it is continuous.
\end{theorem}

\begin{proof}
    The forward direction is obvious because all smooth maps are continuous. Now assume that \mm{\tilde{\phi}} is continuous.
    We first show the following claim. Here let \mm{n,m,s} be the dimensions of \mm{N,M,S} respectively.

    \begin{claim*}
        For any \mm{p\in S}, there is a co-ordinate neighbourhood \mm{U} of \mm{p} such that \mm{i|_U} is an embedding into \mm{N}.
    \end{claim*}
    \begin{proof}
        Pick a co-ordinate neighbourhood \mm{(V,\psi)} around \mm{i(p)} and \mm{(U_1,\tau)} around \mm{p} such that \mm{i(U_1)\subseteq V}.
        Now consider \mm{\tilde{i}:=\psi\circ i\circ \tau^{-1}}. Clearly it is smooth. The fact the \mm{i} is an immersion implies that \mm{\tilde{i}} has a derivative of full rank.
        So, the derivative matrix must then have a \mm{s\times s} minor of full rank. Post compose \mm{\tau} by an appropriate permutation matrix, and redefine \mm{\tilde{i}}
        if necessary, which will ensure that the minor is the first \mm{s\times s} minor. Now extend \mm{\tilde{i}} as follows, $$\tilde{i}_0(\tau_1,..,\tau_s,y_1,y_2,..,y_{n-s}) = (\tilde{i}(\tau_1,..,\tau_s),y_1,y_2,..,y_{n-s})$$
        Clearly \mm{\tilde{i}_0} has an invertible derivative at \mm{\tau(p)}. Thus is a local diffeomorphism on suppose \mm{U}. So, its restriction to the first \mm{s} co-ordinates is an embedding.
        Then \mm{\tau^{-1}(U)} is the desired co-ordinate neighbourhood.
    \end{proof}

    Now fix a point \mm{p\in M}. By the claim above we can find a co-ordinate neighbourhood \mm{(U,\tau)} of \mm{\tilde{\phi}(p)} such that we have an embedding \mm{\tilde{i}:U\to N}.
    Continuity of \mm{\tilde{\phi}} mean that twe can find a co-ordinate ball \mm{(V,\psi)} around \mm{p} such that \mm{\tilde{\phi}(V)\subseteq U}. But, 
    $$\tilde{\phi}(V)\subseteq U\implies i\circ \tilde{\phi}(V)\subseteq i(U)\implies \phi(V)\subseteq i(U)$$
    So, on \mm{V} we have \mm{\tilde{\phi} = i^{-1}\circ \phi}. But \mm{i^{-1}} is a projection map on co-ordinates by the claim above. Thus \mm{\tilde{\phi}} is indeed a smooth map at \mm{p}.
    As \mm{p} was arbitrary, we are done.
\end{proof}

Note also that the claim above is true for any immersion. So, all immersions are local embeddings! Also we note that if \mm{S}
is embedded, restriction of continuous maps is always continuous. Thus restriction of any smooth map to \mm{S} will be smooth.

\begin{corollary*}
    If \mm{M} is an embedded submanifold of \mm{N} and \mm{\phi:S\to N} is a smooth map with \mm{\phi(S)\subseteq M}. Then \mm{\phi} is smooth as a map to \mm{M}.
\end{corollary*}

\begin{theorem}
    (\textbf{Regular value theorem}) Let \mm{\phi:M\to N} be a smooth map. Let \mm{r\in \phi(M)} be a regular value.(i.e, for each \mm{p\in \phi^{-1}(r)} the map \mm{d\phi_p} is full rank)
    Then \mm{\phi^{-1}(r)} is an embedded submanifold of \mm{M} with codimension \mm{dim(N)}.
\end{theorem}
\begin{proof}
    Let \mm{p\in \phi^{-1}(r)}. In any co-ordinate cube, it is evident that the hypothses for constant rank theorem (Rudin) are satisfied.
    thus we can find co-ordinates such that \mm{\phi} is a projection on the first \mm{dim(N)} co-ordinates. This automatically gives us slice charts for \mm{\phi^{-1}(r)}.
\end{proof}

\section{Lie groups}

\begin{definition}
    Let \mm{(G,\cdot)} be an abstract group which is also a manifold. \mm{G} is said to be a \vocab{lie group} if,
    \begin{enumerate}
        \item The map \mm{\cdot:G\times G\to G} is smooth.
        \item The map \mm{()^{-1}:G\to G} is smooth.
    \end{enumerate}
\end{definition}

\begin{exercise*}
    The smoothness of the two maps defined above is equivalent to the smoothness of the single map $$\mu:(x,y)\mapsto xy^{-1}$$.
\end{exercise*}
\begin{proof}
    First let us assume the above two maps are smooth. The note that the map \mm{\tau:G\times G\to G\times G} given by \mm{\tau(x,y) = (x,y^{-1})} as it is the product of two smooth maps.
    Composing it with the product map shows \mm{\mu} is smooth.\\
    Now suppose \mm{\mu} is smooth. Then, its restriction to the submanifold \mm{G':=\{(e,x)|x\in G\}} is smooth. Meaning the inverse map is smooth. Thus the product map which is 
    the \mm{\tau} followed by \mm{\mu} is smooth. This completes the proof.
\end{proof}

It is easy to observe that any lie group is a topological group. Fix \mm{a\in G}. Then we shall call the smooth map
\mm{l_a:G\to G} given by \mm{l_a(g) = a\cdot g} as the left translation and similarly \mm{r_a} defined as \mm{r_a(g) = g\cdot a} is the right translation.
Also note that the inverse map being involutive is a diffeomorphism.

\begin{example}
    We give a list of them!
    \begin{enumerate}
        \item \mm{(\RR^n,+)}
        \item \mm{(\CC^n,+)}
        \item \mm{(\RR^*,\times)}
        \item \mm{(\CC^*,\times)}
        \item \mm{(\mathbb{S}^1,\times)}
        \item \mm{(M_{n\times n}(\RR),+)}
        \item \mm{(M_{n\times n}(\CC),+)}
        \item Any finite dimensional real vector space is a lie group.
        \item Any abstract group is a zero dimansional lie group with discrete topology.
        \item \mm{GL_n(\RR)} is a lie group. It is a manifold because it is an open subset of \mm{M_{n\times n}(\RR)} and the group operations are smooth
    \end{enumerate}
\end{example}

\section{$SL_{n\times n}(\RR)$}
    $SL_{n\times n}(\RR)$ is the group of real matrices of determinant 1. We will now show that it is a lie group.
    But first the following exercise.
    \begin{exercise*}
        Show that the derivative of the determinant map is, $$[\det\,'(A)](H) = \tr(\Adj(A)(H))$$
    \end{exercise*}
    Using this formula, note that if \mm{\det(A)} is 1, then $$\det\,'(A)(A) = \det(A) tr(A^{-1}A) =n \neq 0$$
    Hence the determinant map is a submersion at all points of \mm{SL_{n\times n}(\RR)}. This makes it an embedded submanifold of \mm{GL_{n\times n}(\RR)}.
    Nowe need to check that the \mm{\mu} map is smooth.

    \begin{center}
        \begin{tikzcd}
            \slrg{n}\times \slrg{n} \arrow[r,"i\times i"];\arrow[d,"\mu"] & \glrg{n}\times \glrg{n}\arrow[d,"\tilde{\mu}"]\\
           \slrg{n} \arrow[r,"i"] & \glrg{n}
            
        \end{tikzcd}
    \end{center}
    
    The map \mm{i} is an embedding and hence the map \mm{i\times i} is an embedding. The map \mm{\tilde{\mu}} is smooth(As it is essentially rational in co-ordinates).
    Thus the map \mm{\tilde{\mu}\circ (i\times i)} is smooth. And as determinant is multiplicative, we see that the image of it lies inside the image of \mm{i}.
    Thus by the theorem 9's corollary, the map \mm{\mu} is smooth. This completes the proof.
    
    \begin{exercise*}
        Show that \mm{O_{n\times n}} and \mm{U_{n\times n}} are lie groups.
    \end{exercise*}
    
\end{document}